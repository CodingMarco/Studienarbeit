
%%%
%%% KOMA related
%%%

\usepackage[
    paper=a4paper,
    left=25mm,
    right=25mm,
    top=25mm,
    bottom=20mm,
    footskip=10mm,
    includefoot,
    bindingoffset=0mm
]{geometry}

\usepackage[utf8]{inputenc} % encoding
\usepackage[T1]{fontenc} % selecting font encodings
\usepackage[ngerman]{babel} % German hyphenation
\usepackage[babel,german=quotes]{csquotes} % German quotes
\usepackage{lmodern} %MOST IMPORTANT PACKAGE!!! High resolution font
\renewcommand*\familydefault{\sfdefault} %% für Serifen freie Schrift
\usepackage{textcomp} % für in line {\textmu}

\usepackage{scrlayer-scrpage}
\clearpairofpagestyles
\setkomafont{pageheadfoot}{\sffamily\footnotesize}
\setkomafont{pagehead}{\bfseries}
\setkomafont{pagination}{}

\KOMAoptions{
    headsepline = true,
    footsepline = false,
    plainfootsepline = false
}
\renewcommand*{\sectionmarkformat}{}

\automark{section}
\ihead{\myStudentenname}
\chead{\myDokumententyp}
\ohead{\headmark}
\ofoot*{\pagemark}


\usepackage{scrdate} % Für Datum und Wochentag

%%%
%%% not KOMA related
%%%

\usepackage{relsize}
\def\Cpp{C\kern-.1em\raise.30ex\hbox{\smaller{++}}
\spacefactor1000}  % Für ein schönes "C++", relsize hierfür benötigt.
\usepackage[onehalfspacing]{setspace}

%\usepackage{booktabs} %Never, ever use vertical rules! Never use double rules!
\usepackage{multirow}
\usepackage{comment}

%\usepackage{etoolbox} %use with care! powerful but not can create difficult errors! extends build time
%\AtBeginEnvironment{table}{\sffamily}
%\AtBeginEnvironment{tabular}{\sffamily}

%\usepackage{enumitem}
\usepackage{paralist} % for \begin{compactitem}

% Much much better listings
\usepackage[newfloat, outputdir=aux]{minted}
\newenvironment{code}{\captionsetup{type=listing}}{}
\SetupFloatingEnvironment{listing}{name=Codeausschnitt}

\usepackage{listings}
\lstset{
    basicstyle=\footnotesize, % Global Code Style
    numbers=left,
    numberstyle=\tiny,
    columns=flexible,
    breaklines=true,
    postbreak=\mbox{\textcolor{red}{$\hookrightarrow$}\space},
}
% from https://tex.stackexchange.com/questions/303465/higher-asterisks-in-lstlisting-environment
\makeatletter
\lst@CCPutMacro
    \lst@ProcessOther {"2A}{\raisebox{2pt}{*}}
    \@empty\z@\@empty
\makeatother

\usepackage{etex}
\bibliographystyle{myIEEEtran}

\usepackage{graphicx}
%\usepackage{svg} %no reason to use this (unless you know how to use it)
\usepackage{float}
\usepackage[abs]{overpic} % Für Unterschrift

\usepackage[binary-units = true ]{siunitx}
%\sisetup{scientific-notation = true} % Wissenschaftliche Notation
\sisetup{exponent-product = \cdot}
\sisetup{output-product = \cdot} % Das Punkt statt X
\sisetup{binary-units=true}
\sisetup{output-decimal-marker = {,}}
\sisetup{detect-all} 	% Einstellung das Front für SI gleich Standardfront
\sisetup{per-mode = symbol}
\sisetup{locale=DE}
\usepackage{eurosym}
\DeclareSIUnit{\sieuro}{\mbox{\euro}}

% verwende gerade kein Mathe \usepackage{amsmath, amsthm, amssymb}
\usepackage{amsmath, amssymb}

%\usepackage{booktabs}
\usepackage{graphicx}
\usepackage{adjustbox}
\usepackage{subcaption}
%\usepackage{lscape}
%\usepackage{tikz}
\usepackage{pgfplots}
\pgfplotsset{compat=1.16}
\usepackage{pgfplotstable}

%% Glossar
\usepackage[acronym,nonumberlist,nopostdot]{glossaries}
\newcommand\acrfullr[2][]{\acrshort[#1]{#2} (\acrlong[#1]{#2})}
\usepackage{glossary-mcols} % für Style mcolindex
\glsenablehyper % Verlinkung der Einträge im Text zu Abkürzungen
\setglossarystyle{mcolindex}
\renewcommand{\glossarypreamble}{\glsfindwidesttoplevelname[\currentglossary]} % funktioniert nicht mit mcolindex
\makenoidxglossaries


\usepackage{makecell} % Für multiline Zellen mit \makecell
%\usepackage{tabularx}
\usepackage{longtable}


\usepackage[
    hidelinks,              % Das Links nicht farbig
    bookmarks,              % Bookmarks erstellen
    bookmarksopen,          % Bookmarks bei öffnen des PDF anzeigen
    bookmarksnumbered,      % damit Nummern in Bookmarks
    bookmarksopenlevel = 1, % damit nur die erste Ebene der Bookmarks beim öffnen angezeigt werden
    pdftoolbar=		false,
    pdfmenubar=		false,
    pdftitle={		\myDokumententyp~über~\myTitle~von~\myStudentenname},
    pdfsubject={}
    pdfkeywords={}
    pdfauthor={		\myStudentenname},
    pdfstartpage={	1}
    ]{hyperref}
\addto\extrasngerman{\def\subsectionautorefname{Abschnitt}}
\addto\extrasngerman{\def\subsubsectionautorefname{Abschnitt}}
\addto\extrasngerman{\def\appendixautorefname{Anhang}} % funktioniert auch nicht
\addto\extrasngerman{\def\lstlistingautorefname{Codeausschnitt}}
\renewcommand{\lstlistingname}{Codeausschnitt}

%% Anhang
\usepackage{appendix}
\renewcommand{\appendixname}{Anhang}
\renewcommand{\appendixtocname}{Anhang}
\renewcommand{\appendixpagename}{Anhang}
%\newcommand{\appendixautorefname}{Anhang} % funktioniert nicht
\newcommand*\appendixmore{% see the KOMA-Script documentation
\clearpage
%\addsec{{\appendixname}}%
\renewcommand{\thesubsection}{{\Alph{subsection}}}%
} % damit Appendix subsection mit Buchstaben anfangen

\newcommand{\appendixref}[1]{\hyperref[#1]{Anhang~\ref{#1}}}

\renewcommand{\k}[1]{\textit{#1}}
\newcommand{\f}[1]{\textbf{#1}}
\let\oldautoref\autoref
\let\oldref\ref

\renewcommand{\autoref}[1]{\oldautoref{#1}}
\renewcommand{\ref}[1]{\oldref{#1}}
\newcommand{\fullref}[1]{\autoref{#1} \nameref{#1}}