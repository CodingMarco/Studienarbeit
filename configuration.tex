
%%%
%%% KOMA related
%%%

\usepackage[
    paper=a4paper,
    left=25mm,
    right=25mm,
    top=25mm,
    bottom=20mm,
    includefoot,
    foot=\baselineskip,
    bindingoffset=0mm
]{geometry}

\usepackage[utf8]{inputenc} % encoding
\usepackage[T1]{fontenc} % selecting font encodings
\usepackage[ngerman]{babel} % German hyphenation
\usepackage[babel,german=quotes]{csquotes} % German quotes
\usepackage{lmodern} %MOST IMPORTANT PACKAGE!!! High resolution font
\renewcommand*\familydefault{\sfdefault} %% für Serifen freie Schrift
\usepackage{textcomp} % für in line {\textmu}

\usepackage{scrlayer-scrpage}
\clearpairofpagestyles
\setkomafont{pageheadfoot}{\sffamily\footnotesize}
\setkomafont{pagehead}{\bfseries}
\setkomafont{pagination}{}

\KOMAoptions{
    headsepline = true,
    footsepline = false,
    plainfootsepline = false
}
\renewcommand*{\sectionmarkformat}{}

\automark{section}
\ihead{\myStudentenname}
\chead{\myDokumententyp}
\ohead{\headmark}
\ofoot*{\pagemark}


\usepackage{scrdate} % Für Datum und Wochentag

%%%
%%% not KOMA related
%%%


\usepackage[onehalfspacing]{setspace}

%\usepackage{booktabs} %Never, ever use vertical rules! Never use double rules!
\usepackage{multirow}
\usepackage{comment}

%\usepackage{etoolbox} %use with care! powerful but not can create difficult errors! extends build time
%\AtBeginEnvironment{table}{\sffamily}
%\AtBeginEnvironment{tabular}{\sffamily}

%\usepackage{enumitem}
\usepackage{paralist} % for \begin{compactitem}


\usepackage{listings}
\lstset{
    basicstyle=\footnotesize, % Global Code Style
    numbers=left,
    numberstyle=\tiny,
    columns=flexible,
    breaklines=true,
    postbreak=\mbox{\textcolor{red}{$\hookrightarrow$}\space},
}
% from https://tex.stackexchange.com/questions/303465/higher-asterisks-in-lstlisting-environment
\makeatletter
\lst@CCPutMacro
    \lst@ProcessOther {"2A}{\raisebox{2pt}{*}}
    \@empty\z@\@empty
\makeatother

%\usepackage{color}
%
\definecolor{lightgray}{rgb}{0.95, 0.95, 0.95}
\definecolor{darkgray}{rgb}{0.4, 0.4, 0.4}
%\definecolor{purple}{rgb}{0.65, 0.12, 0.82}
\definecolor{editorGray}{rgb}{0.95, 0.95, 0.95}
\definecolor{editorOcher}{rgb}{1, 0.5, 0} % #FF7F00 -> rgb(239, 169, 0)
\definecolor{editorGreen}{rgb}{0, 0.5, 0} % #007C00 -> rgb(0, 124, 0)
\definecolor{orange}{rgb}{1,0.45,0.13}		
\definecolor{olive}{rgb}{0.17,0.59,0.20}
\definecolor{brown}{rgb}{0.69,0.31,0.31}
\definecolor{purple}{rgb}{0.38,0.18,0.81}
\definecolor{lightblue}{rgb}{0.1,0.57,0.7}
\definecolor{lightred}{rgb}{1,0.4,0.5}
\definecolor{grey}{rgb}{0.9,0.9,0.9}

\newcommand\YAMLcolonstyle{\color{red}\mdseries}
\newcommand\YAMLkeystyle{\color{black}\bfseries}
\newcommand\YAMLvaluestyle{\color{blue}\mdseries}
\lstdefinelanguage{YAML}
{
  keywords={true,false,null,y,n},
  keywordstyle=\color{darkgray}\bfseries,
  ndkeywords={},
  ndkeywordstyle=\color{black}\bfseries,
  identifierstyle=\color{black},
  sensitive=false,
  %moredelim=[l]{}{:},
  %comment=[l]{#},
  morecomment=[s]{/*}{*/},
  commentstyle=\color{purple}\ttfamily,
  stringstyle=\color{blue}\ttfamily,
  morestring=[b]',
  morestring=[b]",
}


% \lstdefinelanguage{YAML}
% {
%   keywords={true,false,null,y,n},
%   keywordstyle=\color{darkgray}\bfseries,
%   basicstyle=\YAMLkeystyle,                                 % assuming a key comes first
%   sensitive=false,
%   comment=[l]{\#},
%   morecomment=[s]{/*}{*/},
%   commentstyle=\color{purple}\ttfamily,
%   stringstyle=\YAMLvaluestyle\ttfamily,
%   moredelim=[l][\color{orange}]{\&},
%   moredelim=[l][\color{magenta}]{*},
%   moredelim=**[il][\YAMLcolonstyle{:}\YAMLvaluestyle]{:\ },   % switch to value style at :
%   morestring=[b]',
%   morestring=[b]",
%   literate =    {---}{{\ProcessThreeDashes}}3
%                 {>}{{\textcolor{red}\textgreater}}1     
%                 {|}{{\textcolor{red}\textbar}}1 
%                 {\ -\ }{{\mdseries\ -\ }}3,
% }




% CSS
\lstdefinelanguage{CSS}{
  keywords={color,background-image:,margin,padding,font,weight,display,position,top,left,right,bottom,list,style,border,size,white,space,min,width, transition:, transform:, transition-property, transition-duration, transition-timing-function,background},	
  sensitive=true,
  morecomment=[l]{//},
  morecomment=[s]{/*}{*/},
  morestring=[b]',
  morestring=[b]",
  alsoletter={:},
  alsodigit={-}
}

% JavaScript
\lstdefinelanguage{JavaScript}{
  morekeywords={typeof, new, true, false, catch, function, return, null, catch, switch, var, if, in, while, do, else, case, break, function},
  morecomment=[s]{/*}{*/},
  morecomment=[l]//,
  morestring=[b]",
  morestring=[b]'
}

\lstdefinelanguage{HTML5}{
    language=html,
    sensitive=true,	
    alsoletter={<>=-},	
    morecomment=[s]{<!-}{-->},
    tag=[s],
    otherkeywords={
    % General
    >,
    % Standard tags
        <!DOCTYPE,
    </html, <html, <head, <title, </title, <style, </style, <link, </head, <meta, />,
        % body
        </body, <body,
        % Divs
        </div, <div, </div>,
        % Paragraphs
        </p, <p, </p>,
        % scripts
        </script, <script,
    % More tags...
    <canvas, /canvas>, <svg, <rect, <animateTransform, </rect>, </svg>, <video, <source, <iframe, </iframe>, </video>, <image, </image>, <header, </header, <article, </article
    },
    ndkeywords={
    % General
    =,
    % HTML attributes
    charset=, src=, id=, width=, height=, style=, type=, rel=, href=, class=, onload=,
    % SVG attributes
    fill=, attributeName=, begin=, dur=, from=, to=, poster=, controls=, x=, y=, repeatCount=, xlink:href=,
    % properties
    margin:, padding:, background-image:, border:, top:, left:, position:, width:, height:, margin-top:, margin-bottom:, font-size:, line-height:, background:, text-align:,
        % CSS3 properties
    transform:, -moz-transform:, -webkit-transform:,
    animation:, -webkit-animation:,
    transition:,  transition-duration:, transition-property:, transition-timing-function:,
    }
}

\lstdefinestyle{htmlcssjs} {%
  % General design
%  backgroundcolor=\color{editorGray},
  basicstyle={\footnotesize\ttfamily},
  frame=b,
  % line-numbers
  xleftmargin={0.75cm},
  numbers=left,
  stepnumber=1,
  firstnumber=1,
  numberfirstline=true,	
  % Code design
  identifierstyle=\color{black},
  keywordstyle=\color{blue}\bfseries,
  ndkeywordstyle=\color{editorGreen}\bfseries,
  stringstyle=\color{editorOcher}\ttfamily,
  commentstyle=\color{brown}\ttfamily,
  % Code
  language=HTML5,
  alsolanguage=JavaScript,
  alsodigit={.:;},	
  tabsize=2,
  showtabs=false,
  showspaces=false,
  showstringspaces=false,
  extendedchars=true,
  breaklines=true,
  % German umlauts
  literate=%
  {Ö}{{\"O}}1
  {Ä}{{\"A}}1
  {Ü}{{\"U}}1
  {ß}{{\ss}}1
  {ü}{{\"u}}1
  {ä}{{\"a}}1
  {ö}{{\"o}}1
}
%

\usepackage{etex}
% \#TODO: Entscheiden welchen Style
\bibliographystyle{myIEEEtran}
%\bibliographystyle{ieeetr} % ist ohne den Strich
%\bibliographystyle{IEEETran} % ist mit dem Strich, dafür ausführlicher


\usepackage{graphicx}
%\usepackage{svg} %no reason to use this (unless you know how to use it)
\usepackage{float}
\usepackage[abs]{overpic} % Für Unterschrift

\usepackage[binary-units = true ]{siunitx}
%\sisetup{scientific-notation = true} % Wissenschaftliche Notation
\sisetup{exponent-product = \cdot}
\sisetup{output-product = \cdot} % Das Punkt statt X
\sisetup{binary-units=true}
\sisetup{output-decimal-marker = {,}}
\sisetup{detect-all} 	% Einstellung das Front für SI gleich Standardfront
\sisetup{per-mode = symbol}
\sisetup{locale=DE}
\usepackage{eurosym}
\DeclareSIUnit{\sieuro}{\mbox{\euro}}

% verwende gerade kein Mathe \usepackage{amsmath, amsthm, amssymb}
% !sehr langsam  \usepackage[colorinlistoftodos,prependcaption, textsize=footnotesize]{todonotes}



%\usepackage{booktabs}
\usepackage{graphicx}
\usepackage{adjustbox}
\usepackage{subcaption}
%\usepackage{lscape}
%\usepackage{tikz}
\usepackage{pgfplots}
\pgfplotsset{compat=1.16}
\usepackage{pgfplotstable}

%% Glossar
\usepackage[acronym,nonumberlist,nopostdot]{glossaries}
\newcommand\acrfullr[2][]{\acrshort[#1]{#2} (\acrlong[#1]{#2})}
\usepackage{glossary-mcols} % für Style mcolindex
\glsenablehyper % Verlinkung der Einträge im Text zu Abkürzungen
\setglossarystyle{mcolindex}
\renewcommand{\glossarypreamble}{\glsfindwidesttoplevelname[\currentglossary]} % funktioniert nicht mit mcolindex
\makenoidxglossaries


\usepackage{makecell} % Für multiline Zellen mit \makecell
%\usepackage{tabularx}
\usepackage{longtable}


\usepackage[
    hidelinks,              % Das Links nicht farbig
    bookmarks,              % Bookmarks erstellen
    bookmarksopen,          % Bookmarks bei öffnen des PDF anzeigen
    bookmarksnumbered,      % damit Nummern in Bookmarks
    bookmarksopenlevel = 1, % damit nur die erste Ebene der Bookmarks beim öffnen angezeigt werden
    pdftoolbar=		false,
    pdfmenubar=		false,
    pdftitle={		\myDokumententyp~über~\myTitle~von~\myStudentenname},
    pdfsubject={}
    pdfkeywords={}
    pdfauthor={		\myStudentenname},
    pdfstartpage={	1}
    ]{hyperref}
\addto\extrasngerman{\def\subsectionautorefname{Abschnitt}}
\addto\extrasngerman{\def\subsubsectionautorefname{Abschnitt}}
\addto\extrasngerman{\def\appendixautorefname{Anhang}} % funktioniert auch nicht
\addto\extrasngerman{\def\lstlistingautorefname{Codeausschnitt}}
\renewcommand{\lstlistingname}{Codeausschnitt}

%% Anhang
\usepackage{appendix}
\renewcommand{\appendixname}{Anhang}
\renewcommand{\appendixtocname}{Anhang}
\renewcommand{\appendixpagename}{Anhang}
\newcommand{\Appendixautorefname}{Anhang} % funktioniert nicht
 \newcommand*\appendixmore{% see the KOMA-Script documentation
   \clearpage
   %\addsec{{\appendixname}}%
   \renewcommand{\thesubsection}{{\Alph{subsection}}}%
 } % damit Appendix subsection mit Buchstaben anfangen


 \renewcommand{\k}[1]{\textit{#1}}
 \newcommand{\f}[1]{\textbf{#1}}
 \let\oldautoref\autoref
 \let\oldref\ref

 \renewcommand{\autoref}[1]{\oldautoref{#1}}
 \renewcommand{\ref}[1]{\oldref{#1}}
 \newcommand{\fullref}[1]{\autoref{#1} \nameref{#1}}