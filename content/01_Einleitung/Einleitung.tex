\section{Einführung}
\label{sec:Einführung}

\subsection{Problemstellung und Zielsetzung}
\label{sec:Problemstellung}
In der App \emph{Blitzer.de} können Benutzer mobile und teilstationäre Radarkontrollen sowie Staus, Unfälle und sonstige Gefahren melden.
Diese werden anderen Benutzern angezeigt, wodurch sie vorgewarnt sind und sich entsprechend vorsichtig verhalten können.
Jedoch sind die Meldungen naturgemäß zeitverzögert, da aktive Benutzer benötigt werden, um sie aktuell zu halten.
Es lässt sich argumentieren, dass die Zeitverzögerung insbesonders in den frühen Morgenstunden und in ländlichen Gebieten signifikant ist,
da sich hier nur wenige Benutzer auf den Straßen befinden.
Dies ist besonders für mobile Radarkontrollen entscheidend, da solche jeden Tag zu unterschiedlichen Zeiten und an unterschiedlichen Orten auf- und abgebaut werden.

Andererseits ist auch denkbar, dass es Zusammenhänge zwischen vergangenen und zukünftigen Standorten von Radarkontrollen gibt.
Beispielsweise liegt es nahe, dass mobile Radarkontrollen räumlich und zeitlich möglichst gut gestreut werden.
Daher kann man annehmen, dass die Gefahr für eine mobile Radarkontrolle an einem Ort im ländlichen Bereich sehr gering ist, wenn dort am vorherigen Tag eine solche anzutreffen war.

Das Ziel der vorliegenden Arbeit ist daher, die Gefahr für mobile Radarkontrollen anhand historischer Daten und insbesondere derer der letzten Tage für den Folgetag zu prognostizieren.
Wie in \autoref{sec:VerwanteteForschung} gezeigt wird, gibt es einige vielversprechende Ansätze, wie Neuronale Netze (\Acrshortpl{nn}) verwendet werden können, um sehr ähnliche Probleme anzugehen, wie z.B. die Vorhersage von Verbrechen.
Daher sollen auch in dieser Arbeit \Acrshortpl{nn} verwendet werden, um die Vorhersagen anzustellen.

Die erwähnten historischen Daten bestehen aus allen gemeldeten Standorten von mobilen Radarkontrollen der vorhergegangenen Jahre.
Dieser Datensatz wurde freundlicherweise von der Eifrig Media GmbH bereitgestellt.
Die Eifrig Media GmbH steht hinter der Entwicklung der oben genanten App \emph{Blitzer.de}.
Der Datensatz enthält über 7,7 Millionen über die App \emph{Blitzer.de} gemeldete mobile Radarkontrollen vom 22.05.2014 bis zum 25.10.2021.
Daraus ergibt sich ein Zeitraum von 2713 Tagen und somit ca. 21.300 Meldungen pro Tag.
Die Meldungen beschränken sich hierbei auf Deutschland, wobei die Eifrig Media GmbH in \cite{AboutBlitzerDe} angibt, auch äquivalente Angebote zu \emph{Blitzer.de} im Ausland bereitzustellen.
Jeder Eintrag des Datensatzes enthält u.A. die Koordinaten, sowie den ungefähren Aufbau- und Abbauzeitpunkt der Radarkontrolle.
Diese Daten basieren jedoch auf den Angaben der Appbenutzer und sollten daher nicht unkritisch betrachtet werden.
So entsprechen die Koordinaten dem Standort des Appbenutzers zum Zeitpunkt der Meldung in der App.
Da die Meldungen meist bei laufender Fahrt erfolgen, kann die gemeldete Position auf einem Straßenabschnitt relativ weit vor oder nach der tatsächlichen Position der Radarkontrolle liegen.
Der Abbauzeitpunkt basiert ebenfalls auf Angaben der Appbenutzer.
Sobald ein Benutzer an einer eingetragenen Radarkontrolle vorbeigefaren ist wird dieser gefragt, ob die Radarkontrolle gesehen wurde.




\subsection{Aufbau der Arbeit}
\label{sec:Aufbau}



