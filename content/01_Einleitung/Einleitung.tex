\section{Einführung}
\label{sec:Einführung}

\subsection{Problemstellung und Zielsetzung}
\label{sec:Problemstellung}
In der App \emph{Blitzer.de} können Benutzer mobile und teilstationäre Radarkontrollen sowie Staus, Unfälle und sonstige Gefahren melden.
Diese werden anderen Benutzern angezeigt, wodurch sie vorgewarnt sind und sich entsprechend vorsichtig verhalten können.
Jedoch sind die Meldungen naturgemäß zeitverzögert, da aktive Benutzer benötigt werden, um sie aktuell zu halten.
Es lässt sich argumentieren, dass die Zeitverzögerung insbesonders in den frühen Morgenstunden und in ländlichen Gebieten signifikant ist,
da sich hier nur wenige Benutzer auf den Straßen befinden.
Dies ist besonders für mobile Radarkontrollen entscheidend, da solche jeden Tag zu unterschiedlichen Zeiten und an unterschiedlichen Orten auf- und abgebaut werden.

Andererseits ist auch denkbar, dass es Zusammenhänge zwischen vergangenen und zukünftigen Standorten von Radarkontrollen gibt.
Beispielsweise liegt es nahe, dass mobile Radarkontrollen räumlich und zeitlich möglichst gut gestreut werden.
Daher kann man annehmen, dass die Gefahr für eine mobile Radarkontrolle an einem Ort im ländlichen Bereich sehr gering ist, wenn dort am vorherigen Tag eine solche anzutreffen war.

Das Ziel der vorliegenden Arbeit ist daher, die Gefahr für mobile Radarkontrollen anhand vergangener Daten für den Folgetag zu prognostizieren.
Die erwähnten vergangenen Daten bestehen aus allen Standorten von mobilen Radarkontrollen der vorhergegangenen Jahre.




\subsection{Aufbau der Arbeit}
\label{sec:Aufbau}



