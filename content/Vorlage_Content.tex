\section{Vorlage}
\subsection[USB]{\acrshort{usb}}
\acrfull{usb} ist ein Schnittstelle zur Anbindung von \acrshort{usb}-Geräten.


\subsubsection{Zitieren} \label{sssec:zitieren}

Zitieren \cite[s. 31]{usb_developer_guide} angeschlossen werden, die sich auf Stern-Sternnetzten, mit zu 7 Ebenen \cite[4.1.1]{usb_developer_guide}, aufteilen können. Der Steckplatz von einem Hub zum Host, als upstream port bezeichnet, Buchsen für Peripheriegeräten oder weiter Hubs als downstream ports.
Die Struktur einer logischen Verbindung von Host zu einem Slave ist in \autoref{fig:usb_pipe_modell} 
\begin{figure}[h]
    \centering
    \includegraphics[width=0.75\textwidth,height=6cm,keepaspectratio=true]{content/images/Cypress_ubs_figure_1.PNG}
    \caption{\acrshort{usb} pipe Modell \cite[Figure~1]{usb_developer_guide}}
    \label{fig:usb_pipe_modell}
\end{figure}
\par
Ein endpoints kann entweder die erwähnten Protokollfunktionen bereitstellen, Daten vom Host zu empfangen, dann ist dies ein IN endpoint, oder Daten für den Host bereitzustellen dann ist dies ein OUT endpoint. Diese Terminologie folgt der in der USB Spezifikation verwendete Betrachtung des Bus aus Sicht des Hosts.
Neben der Richtung eines endpoints, IN oder OUT, werden vier Übertragungsarten \cite[5.4]{usb_developer_guide} unterschieden. Die Übertragungsart wird durch die Gerätekonfiguration für jeden endpoint festleget und wird in \autoref{sssec:endpoints} weiter ausgeführt. Die Funktionen der Übertragungsarten sind \cite[4.7]{usb_developer_guide}:

\begin{compactitem}
    \item \textbf{control transfer} Für das Übertragen von Konfigurations- und Steuerungsbefehlen über die control pipe.
    \item \textbf{interrupt transfer} Für die Übertragung kleiner Datenmengen mit einer möglichst kleinen Latenz über eine data pipe .
    \item \textbf{bulk transfer} Für die Übertragung grö{\ss}er Datenmengen ohne zugesicherten Geschwindigkeit oder Latenz über eine data pipe ohne Datenverluste.
    \item \textbf{isochronous transfer}  Für Übertragungen mit zugesicherten Übertragungskapazität über eine data pipe ohne Fehlerbehebung.
\end{compactitem}
Welche Übertragungsarten ein Gerät unterstützt, und damit über welche endpoints es verfügt kann frei gewählt werden oder kann einer %\acrshort{ubs} Klassen Spezifikation folgen. Eine solche Spezifikation definiert neben dem Aufbau der endpoints weiter einheitliche Funktionen damit Geräte, einer gleichen Klasse, problemlos ausgetauscht werden können.

Softwareseitig wird die Übertragung in drei Ebenen unterteilt, diese sind in \autoref{fig:usb_schnittstellen_abstraktion} abgebildet.
\begin{figure}[h]
    \centering
    \includegraphics[width=.75\textwidth,keepaspectratio=true]{content/images/usb_spec_figure_5_2.PNG}
    \caption{\acrshort{usb} Schnittstellen Abstraktion \cite[Figure~5-2]{usb_developer_guide}}
    \label{fig:usb_schnittstellen_abstraktion}
\end{figure}



\subsubsection{Endpoints} \label{sssec:endpoints}
Nach  \acrshort{usb} Spezifikation ist ein endpoint eine eindeutiger adressierbarer Schnittstelle eines \acrshort{usb} Gerätes der Daten senden oder empfangen werden. Ausgenommen ist der control endpoint, dessen Verwendung wird in \autoref{sssec:Konfigurationsvorgang} ausführlicher beschrieben.

Die \acrshort{usb} Spezifikation definiert vier Arten von endpoints mit der maximalen Paketgrö{\ss}e und den unterstützen Übertragungsgeschwindigkeiten. Die Art eines endpoints definiert gleichzeitig die die \k{transfer}-Art aus \fullref{sssec:usb_ueberblick} und sind \cite[Kapitel~8]{usb_developer_guide}:
\begin{compactitem}
    \item \textbf{control endpoint} für control Transfers und muss in jedem Gerät vorhanden sein, um Geräteinformation zu übertragen. Die Übertragung sind fehlerbehebend und können, bei High-Speed, bis zu 10\% der Buslast einnehmen.
    \item \textbf{control endpoint} für control Transfers und muss in jedem Gerät vorhanden sein, um Geräteinformation zu übertragen. Die Übertragung sind fehlerbehebend und können, bei High-Speed, bis zu 10\% der Buslast einnehmen.
\end{compactitem}

Im  device descriptor werden allgemeine Informationen für ein Gerät bereitgestellt, die Datenfelder sind in \autoref{tab:ubs_device_descriptor} aufgetragen.
\begin{table}[h]
    \centering
    \begin{tabular}{|l|l|}
        \hline
        \textbf{Feld}      & \textbf{Bedeutung}                                 \\
        \hline
        bLength            & Länge des descriptor               \\
        \hline
        bDescriptorType    & descriptor Type, hier DEVICE (0x01)                 \\
        \hline
        bcdUSB             & \acrshort{usb} Spezifikationsversion als \acrshort{bcd}                   \\
        \hline
        bDeviceClass       & device class                                       \\
        \hline
        bDeviceSubClass    & device subclass                                    \\
        \hline
        bDeviceProtocol    & Device Protocol                                    \\
        \hline
        bMaxPacketSize     & Maximale Packet Größe für endpoint 0               \\
        \hline
        idVendor           & \acrfull{vid}, vergeben durch \acrshort{usbif})        \\
        \hline
        idProduct          & \acrfull{produkt_id}, vergeben durch den Hersteller   \\
        \hline
        bcdDevice          & Geräteversion als \acrshort{bcd}                              \\
        \hline
        iManufacturer      & Position der Herstellerbezeichnung in Packet       \\
        \hline
        iProduct           & Position der Gerätebezeichnung in Packet           \\
        \hline
        iSerialNumber      & Position der Seriennummer in Packet                \\
        \hline
        bNumConfigurations & Anzahl der unterschiedlichen Gerätekonfiguration   \\
        \hline
    \end{tabular}
    \caption{device descriptor}
    \label{tab:usb_device_descriptor} % #TODO Offset und Size entfernen
\end{table}

