\subsubsection{Räumlich-zeitliche Vorhersagen}
\label{sec:STPredictions}
Im Themenfeld der räumlich-zeitlichen Vorhersagen (engl. spatio-temporal predictions) gibt es viele verschiedene Anwendungsfälle.
Zu diesen Anwendungsfällen gehört die Vorhersage von Niederschlägen, Verbrechen, Verkehrsunfällen, Verkehrsaufkommen sowie der Bedarf an Taxis \cite{ConvLSTM,CrimeConvLSTM,CrimeSTResNet,HeteroConvLSTM,TrafficVolumeGraphDCRNN,STResNetOriginal}.
So unterschiedlich die Themengebiete auch sind, lässt sich nach \cite{DLTraff} dennoch eine gemeinsame, formale Problemstellung definieren.
Wenn die Daten als Raster vorliegen, lassen sie sich laut Jiang et al. als 4D-Tensor $\mathbb{R}^{T \times H \times W \times C}$ beschreiben.
Dabei ist $T$ die Anzahl der Zeitschritte, $H \times W$ die räumliche Dimension (Höhe mal Breite), und $C$ die Anzahl an Kanälen.
Besteht der Datensatz beispielsweise aus 100 \acrshort{rgb}-Bildern mit einer Größe von 28x28 Pixeln, kann er mit dem Tensor $\mathbb{R}^{100 \times 28 \times 28 \times 3}$ beschrieben werden.
Die Problemstellung lässt sich nun als die in \autoref{eq:FormalSTPrediction} gezeigte Abbildung $f$ formulieren.

\begin{equation}
    f: [X_{t-(\alpha-1)}, \dots, X_{t-1}, X_t] \to [X_{t+1}, X_{t+2}, \dots, X_{t+\beta}]
\label{eq:FormalSTPrediction}
\end{equation}

Dabei repräsentiert $X_t \in \mathbb{R}^{H \times W \times C}$ ein Raster zum Zeitpunkt $t$.
Außerdem ist $\alpha$ die Anzahl an Zeitschritten, anhand derer die Vorhersage getroffen wird und $\beta$ die Anzahl an Zeitschritten, die vorhergesagt werden soll.
Es ist auch möglich, nur einen Zeitschritt vorherzusagen.
Wenn beispielsweise anhand von 16 vergangenen Tagen der 17. Tag vorhergesagt werden soll, dann ist $\alpha = 16$ und $\beta = 1$.
Daraus ergibt sich $f: [X_{t-15}, X_{t-14}, \dots, X_{t-1}, X_t] \to X_{t+1}$.
Mit $f$ ist nun die Funktion definiert, die durch ein \acrshort{nn} approximiert werden soll.
Für ein solches \acrshort{nn} existieren verschiedene Architekturen.
Im von Jiang et al. verfassten Paper \cite{DLTraff} befindet sich ein detaillierter Vergleich vieler dieser Architekturen anhand ihrer Ansätze und Performance.
Einige Beispiele werden im folgenden zusammengefasst.

\textbf{ST-ResNet} wird in \cite{STResNetOriginal} definiert und ist eines der populärsten Modelle für räumlich-zeitliche Vorhersagen.
Jiang et al. machen dies an der Anzahl der Paper fest, die Referenzen auf das eben erwähnte Paper enthalten.
Der Anwendungsfall für den ST-ResNet konzipiert wurde, ist die Vorhersage der Bewegung von Menschenmassen.
Dafür wird nach der Raum in Rasterzellen unterteilt.
Pro Zeitschritt sind jeder Zelle zwei Werte zugeordnet: die Anzahl an Menschen, die die Zelle betreten und die Anzahl an Menschen, die die Zelle verlassen.
Dies ist ein Beispiel für ein Datensatz mit zwei Kanälen.
