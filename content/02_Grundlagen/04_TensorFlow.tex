\subsection{TensorFlow}
\label{sec:TensorFlow}

TensorFlow ist eine Python-Bibliothek für viele verschiedene Machine Learning Aufgaben.
TensorFlow implementiert dafür u.A. mathematische Grundfunktionen wie Tensoroperationen oder Gradientenberechnung.
Da sehr viel Fachwissen nötig wäre um alleine mit TensorFlow \acrshortpl{nn} zu implementieren, wird mit TensorFlow wird auch die Keras-Bibliothek ausgeliefert.
Keras baut auf TensorFlow auf und bietet ein High-Level-Framework für neuronale Netze.
Mit Keras können \acrshortpl{nn} relativ einfach definiert, trainiert und ausgewertet werden.
Keras bietet sehr viele vordefinierte Layer, darunter Feedforward-, \acrshort{lstm}-, \acrshort{cnn}- sowie MaxPooling-Layer.
Bei Bedarf können jedoch auch auch eigene Layer definiert werden.
Aus diesen vordefinierten und ggf. benutzerdefinierten Layern können beliebig komplexe Modelle zusammengesetzt werden.
Außerdem kann der Ablauf der Trainierung dieser Modelle komplett angepasst werden, inklusive des Optimierers und der Verlustfunktion.
Da die Trainierung von \acrshortpl{nn} sehr rechenintensiv ist, bietet TensorFlow die Möglichkeit, die Berechnungen auf einer unterstützten Grafikkarte auszuführen.
Da Grafikkarten Berechnungen hochgradig parallelisiert ausführen, können Modelle deutlich schneller trainiert werden als auf der CPU.
In [TODO: Anhang] befindet sich ein Geschwindigkeitsvergleich zwischen CPU und Grafikkarte.
