\subsection{Faltungsnetze}
\label{sec:CNN}

Mit der bisher vorgestellten \acrshort{lstm}-Architektur können die zeitlichen Zusammenhänge bei der Vorhersage von mobilen Radarkontrollen erlernt werden.
Jedoch wurde bereits motiviert, warum auch räumliche Zusammenhänge für diese Aufgabenstellung von Bedeutung sind.
Prinzipiell ist es möglich, 2D-Daten mit Feedforward-Netzen zu verarbeiten.
Dazu müssen die 2D-Daten verflacht werden.
Stellt man sich ein Zweidimensionales Bild vor bedeutet das, dass jedes Pixel einem Eingangsneuron zugeführt wird.
Hierbei geht die räumliche Struktur der Daten jedoch verloren - das \acrshort{nn} kann nicht wissen, welche Pixel in alle vier Richtungen benachbart waren.
Mit genug Trainingsdaten können einfache Klassifizierungsaufgaben dennoch mit dieser Netzarchitektur bearbeitet werden.
Chollet demonstriert beispielsweise in \cite[S. 53]{DeepLearningPythonKeras}, dass bei der Klassifizierung von handgeschriebenen Ziffern mit einem Feedforward-Netz eine Korrektklassifizierungsrate von 97,8\% erreicht werden kann.
Das Problem mit Feedforward-Netzen ist jedoch, dass sie nur globale Muster erlernen, also beispielsweise eine Ziffer als ganzes.
Ist dieselbe Ziffer nun etwas im Bild verschoben oder auf sonstige weise verzerrt, wird sie von einem Feedforward-Netz nicht mehr erkannt.
Dies kann dadurch visualisiert werden, dass nach einer Verzerrung die einzelnen Pixel an ganz anderen Eingangsneuronen anliegen.

Abhilfe hierbei verschaffen Faltungsnetze (engl. \acrlongpl{cnn}, \acrshortpl{cnn}).
\acrshortpl{cnn} entsprechen dem Stand der Technik, wenn es um die Verarbeitung von 2D-Daten geht.
Wenn das vorherige Beispiel nochmals betrachtet wird, kann man erkennen, welche Eigenschaften der Ziffern sich nicht durch eine Verzerrung ändern: die Zusammensetzung der Ziffern aus kleineren Merkmalen.
Selbst wenn eine Ziffer verschoben oder leicht verzerrt wird, werden sich relativ zueinander immer noch die selben Linien an den selben Stellen kreuzen oder berühren.
Die Betrachtung von 2D-Daten als Zusammensetzung von lokalen Mustern wird in \autoref{fig:LokaleMuster} verdeutlicht.

\begin{figure}[h]
    \centering
    \includegraphics[width=0.4\textwidth,height=4cm,keepaspectratio=true]{content/images/LokaleMuster.png}
    \caption{Lokale Muster wie Ränder und Linien in einer handgeschriebenen Ziffer \cite[Abb. 5.1]{DeepLearningPythonKeras}}
    \label{fig:LokaleMuster}
\end{figure}

\acrshortpl{cnn} erkennen nach \cite[S. 164]{DeepLearningPythonKeras} nun eben diese lokalen Muster.
Chollet geht in \cite{DeepLearningPythonKeras} auf Seite 165 auf die daraus resultierenden Eigenschaften von \acrshortpl{cnn} ein.
Zunächst sei die Erkennung der lokalen Muster nach Chollet translationsinvariant.
Dies bedeute, dass die Muster an beliebigen Stellen im Bild erkannt werden können.
Außerdem könne durch das Hintereinanderschalten von mehreren \acrshort{cnn}-Schichten erreicht werden, dass Hierarchien von Mustern erlernt werden.
Aus diesen beiden Eigenschaften folgt, dass es für ein \acrshort{cnn} keinen Unterschied macht, ob die zu erkennenden globalen Muster (beispielsweise eine Ziffer) in der Eingabe verschoben sind.

Als nächstes stellt sich die frage, wie \acrshortpl{cnn} lokale Muster erkennen können.
Dies wird durch die Faltungsoperation erreicht.
Die Idee der Faltungsoperation ist nach \cite{6S191CNN}, einen sogenannten Filter zu verwenden, der lokale Muster erkennt.
Dieser zweidimensionale Filter wird über die 2D-Eingabe "`geschoben"' und erkennt somit, wo sich in der Eingabe ein bestimmtes Muster befindet.
Mathematisch gesehen ist ein Filter eine quadratische Matrix.
Die Elemente der Matrix sind die erlernbaren Gewichtungen, die je nach ihren Werten verschiedene Muster erkennen.
Die Erkennung geschieht dadurch, dass der Filter an der jeweiligen Stelle komponentenweise mit der Eingabe multipliziert und die einzelnen Werte anschließend aufsummiert werden.
Diese Berechnung wird in \autoref{fig:ConvExample} verdeutlicht.

\begin{figure}[h]
    \centering
    \includegraphics[width=1.0\textwidth,height=8cm,keepaspectratio=true]{content/images/ConvExample.png}
    \caption{Ein Schritt der Faltung des Filters mit der Eingabe \cite{6S191CNN}}
    \label{fig:ConvExample}
\end{figure}

Links oben in der Abbildung ist der Filter dargestellt.
Dieser Filter erkennt schräge Linien von links oben nach rechts unten.
Bei einer perfekten Übereinstimmung wie im Beispiel der Abbildung wird das Ergebnis der Berechnung maximal.
Angenommen, das Pixel oben in der Mitte der Eingabe wäre $1$, dann wäre das Ergebnis der Berechnung nur $7$.
Der Wert des Ergebnisses ist also ein Maß für die Übereinstimmung des Filters mit der Eingabe.
Auf diese Ergebnis wird in der Praxis nach \cite{6S191CNN} noch eine Aktivierungsfunktion angewendet, wofür meistens \emph{\acrshort{relu}} verwendet wird.
Negative Pixel werden dadurch zu null.
Wird der Filter nun mit einer Schrittweite von $1$ in $x$- und $y$-Richtung über die gesamte Eingabe geschoben und die Berechnung jedes mal ausgeführt, entsteht eine sogenannte Merkmalskarte (engl. feature map).
Die Merkmalskarte gibt an, wo im Bild die Übereinstimmung mit dem durch den Filter beschriebenen Muster wie groß ist.
Damit die Merkmalskarte die gleiche Höhe und Breite wie die Eingabe hat, kann nach \cite[S. 168 f.]{DeepLearningPythonKeras} die Eingabe vor der Faltungsoperation rundherum mit nullen aufgefüllt werden.
Dies wird auch Padding genannt.


