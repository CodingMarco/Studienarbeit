\subsection{Verwandte Anwendungsfälle}
\label{sec:VerwandteAnwendungsfaelle}
In \autoref{sec:STPredictions} wurden bereits einige Anwendungsfälle von räumlich-zeitlichen Vorhersagen genannt.
In diesem Abschnitt sollen diese Anwendungsfälle im Bezug auf ihre Relevanz für die Vorhersage von mobilen Radarkontrollen genauer untersucht werden.

Anhand von \autoref{tab:STStructureUseCases} können die verschiedenen Anwendungsfälle schon grob in Kathegorien eingeteilt werden.
An der Tabelle ist außerdem zu erkennen, dass vor allem die Vorhersage von Verkehrsunfällen und Verbrechen eine große Relevanz für die vorliegende Problemstellung hat, da es sich dort ebenfalls um räumlich kontinuierliche Ereignisse handelt, die in unregelmäßigen Zeitabständen auftreten.
Eine weitere Eigenschaft dieser Anwendungsfälle ist, dass die Ereignisse nur sehr sporadisch auftreten.
Während Wetterstationen beispielsweise alle paar Sekunden einen Messwert liefern, treten pro Tag vergleichsweise wenige Verkehrsunfälle und Verbrechen pro Tag auf.
Bei mobilen Radarkontrollen verhält es sich ähnlich.
Beispielsweise werden in einem 50x50 km Bereich um Stuttgart herum im Durchschnitt pro Tag xxxTODOxxx mobile Radarkontrollen gemeldet.



%   - Niederschlagsvorhersage als weiter entfernt
%   - Verbrechensvorhersage als sehr ähnlichen Anwendungsfall
