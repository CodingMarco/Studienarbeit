\subsection{Rasterisierung}
\label{sec:Rasterisierung}
Um ein Rasterbasiertes Modell wie ConvLSTM oder ST-ResNet für die Vorhersage von mobilen Radarkontrollen verwenden zu können, müssen die räumlich und zeitlich kontinuierlichen Datenpunkte aus dem Datensatz zunächst rasterisiert werden.
Dazu ist es sinnvoll nochmals zu betrachten, wie die Daten ursprünglich vorliegen.
Jede Meldung ist eine Zeile in einer Datenbanktabelle mit den Spalten \emph{Längengrad}, \emph{Breitengrad}, \emph{Aufbauzeitpunkt} und \emph{Abbauzeitpunkt}.
Das Ziel besteht darin, die Datenpunkte so zu verarbeiten, dass pro Zeitraum $T$ ein Raster aus $n \times n$ quadratischen Zellen mit der Seitenlänge $k$ entsteht, bei dem jeder Zelle ein Wert $x$ zugeordnet ist, der etwas über die gemeldeten Radarkontrollen in dieser Zelle aussagt.
Die Werte $T,~n,~k$ sind hierbei variable Hyperparameter und müssen möglichst sinnvoll ausgewählt werden.
Außerdem muss festgelegt werden, nach welcher Regel der Wert $x$ berechnet werden soll.
Bei der Auswahl des Zeitraums $T$ muss zwischen der gewünschten zeitlichen Genauigkeit und der Anzahl an verfügbaren Datenpunkten abgewägt werden.
Wird $T$ beispielsweise zu klein gewählt, enthält der betrachtete Bereich zu wenige Meldungen, um auf deren Basis aussagekräftige Vorhersagen treffen zu können.
Wird $T$ hingegen zu groß gewählt (wie z. B. eine Woche oder ein Monat), haben die Vorhersagen weniger praktischen Nutzen.
Außerdem könnten dann insgesamt zu wenige Zeitschritte vorliegen, um das Modell trainieren zu können.
Nach einigen Versuchen scheinen 24 Stunden für $T$ ein guter Wert zu sein, da somit der praktische Nutzen erhalten bleibt und es dennoch genug Meldungen pro Zeitschritt gibt.

Als nächstes sind $k$ udn $n$ zu wählen.
Die Seitenlänge des gesamten betrachteten Bereichs berechnet sich daraus mit $n \cdot k$.
Bei der Wahl von $k$ muss dieselbe Abwägung getroffen werden wie bei der Wahl von $T$, jedoch geht es hier um den praktischen Nutzen der Vorhersagen je nach räumlicher Auflösung.
Wird $k$ zu groß gewählt, ist die zeitliche Auflösung zu gering, als dass die Vorhersagen nützlich wären.
Wenn $k$ hingegen zu klein gewählt wird, gibt es in zu wenigen Zellen eine Meldung, um das Modell gut trainieren zu können.
Bei der Wahl von $k$ kommt es außerdem darauf an, ob eher städtischer oder ländlicher Raum betrachtet werden soll.
Für den ländlichen Raum scheint $k = 4~\text{km}$ eine gute Wahl zu sein, da die Radarkontrollen dort nicht sonderlich dicht liegen.
Für den städtischen Raum ist dieser Wert jedoch eher zu groß, da sich in einem 4 km Radius vergleichsweise deulich mehr Straßen befinden.
Außerdem ist die Dichte an Radarkontrollen im städtischen Raum höher, weshalb ein geringeres $k$ von einem oder zwei Kilometer durchaus möglich ist.
Wird (beispielsweise in Baden-Württemberg) insgesamt ein größerer Raum abgedeckt, besteht die meiste Fläche aus ländlichem Gebiet.
Daher wird im folgenden mit $k = 4~\text{km}$ fortgefahren.
Die Wahl von $n$ beeinflusst nun den insgesamt durch das Raster abgedeckten Raum.
Optimalerweise sollte dieser Wert möglichst groß sein, um möglichst viele Trainingsdaten einzuschließen.
Andererseits ist $n$ durch den währen des Trainings verfügbaren Speicher begrenzt.
Wird das Training auf einer Grafikkarte ausgeführt, muss das komplette Modell und alle Trainingsdaten, die pro Batch verarbeitet werden in den Grafikspeicher passen.
Da der Speicherbedarf quadratisch mit $n$ ansteigt, ist man hier in der Praxis sehr eingeschränkt.
Bei einem Grafikspeicher von 8 GB und einer Batch Size von 12 ist $n = 50$ gut möglich.
Mit $k = 4~\text{km}$ ergibt sich dann eine Seitenlänge des gesamten Rasters von 200 km.

Für die Berechnung von $x$ gibt es mehrere Möglichkeiten.
Zunächst könnte $x = 1$ gesetzt werden, wenn sich in der Zelle mindestens eine Radarkontrolle befindet, ansonsten $x = 0$.
Alternativ kann $x$ auf die Anzahl der Radarkontrollen gesetzt werden, die sich im jeweiligen Zeitraum in der Zelle befinden.
Dies hat jedoch den Nachteil, dass eine Radarkontrolle mit einer kurzen Standdauer gleich gewichtet wird wie eine Radarkontrolle mit einer deutlich längeren Standdauer.
Daher bietet es sich an, die Standdauer aller Radarkontrollen in einer Zelle aufzusummieren.
Mit dieser Möglichkeit enthält $x$ am meisten Informationen über die Radarkontrollen in der jeweiligen Zelle.
Außerdem kann diese Repräsentation auch nach der Rasterisierung einfach in eine binäre Unterscheidung umgewandelt werden, indem pro Zelle $x := x > 0$ gesetzt wird.

Nun stellt sich die Frage, wie die Rasterisierung mit den erörterten Parametern möglichst effizient ausgeführt werden kann.
Die einfachste Möglichkeit besteht darin, alle Meldungen pro Rasterzelle und Datum nacheinander mit einem Python-Skript von der MariaDB-Datenbank abzufragen.
Pro Rasterzelle dauert diese Abfrage jedoch ca. 0,32 Sekunden.
Bei 200 Zellen und 2648 Tagen würde die Rastergenerierung damit jedoch 47 Stunden dauern, was unpraktikabel ist.
Eine mögliche Optimierung ist, eine Geodatenbank zu verwenden, die das Filtern nach Koordinaten effizienter ausführen kann als MariaDB.
Eine solche Geodatenbank ist PostgreSQL mit der Erweiterung PostGIS.
Die Abkürzung \acrshort{gis} in PostGIS steht für \acrlong{gis}.
Dies ist ein Begriff für Systeme, die besonders auf die Verarbeitung von räumlichen Daten optimiert sind, wozu auch Geodatenbanken gehören.
Eines der wichtigsten Features einer Geodatenbank ist ein räumlicher Index (engl. spatial index).
Ein solcher ist laut \cite[S. 2707 f.]{SpatialIndexingTechniques} eine Datenstruktur, die einen schnelleren Zugriff auf räumliche Daten bietet als traditionelle eindimensionale Indexe wie ein B\textsuperscript{+}-Baum.


