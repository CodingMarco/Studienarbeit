\subsection{Definition des ConvLSTM-Modells in TensorFlow}
\label{sec:ModellDefTF}

\begin{code}
\begin{minted}[
    linenos,
    numbersep=10pt,
    gobble=0,
    frame=lines,
    framesep=2mm]{python}
def get_model(input, weights):
    model = keras.models.Sequential([
        input,
        layers.ConvLSTM2D(
            filters=64, kernel_size=(3, 3),
            padding="same", return_sequences=True, activation="relu",
        ),
        layers.MaxPooling3D(pool_size=(2, 1, 1), strides=(2, 1, 1)),
        layers.ConvLSTM2D(
            filters=64, kernel_size=(3, 3),
            padding="same", return_sequences=True, activation="relu",
        ),
        layers.MaxPooling3D(pool_size=(2, 1, 1), strides=(2, 1, 1)),
        layers.ConvLSTM2D(
            filters=32, kernel_size=(3, 3),
            padding="same", return_sequences=True, activation="relu",
        ),
        layers.MaxPooling3D(pool_size=(2, 1, 1), strides=(2, 1, 1)),
        layers.ConvLSTM2D(
            filters=32, kernel_size=(3, 3),
            padding="same", return_sequences=True, activation="relu",
        ),
        layers.MaxPooling3D(pool_size=(2, 1, 1), strides=(2, 1, 1)),
        layers.Conv2D(
            filters=1, kernel_size=(3, 3), activation="sigmoid", padding="same"
        )
    ])

    loss_fn = weighted_binary_crossentropy(weights)

    model.compile(
        loss=loss_fn, optimizer=keras.optimizers.Adam(),
        metrics=METRICS
    )

    return model, loss_fn
\end{minted}
\captionof{listing}{Definition des ConvLSTM-Modells in TensorFlow}
\label{lst:ModellDefTF}
\end{code}
