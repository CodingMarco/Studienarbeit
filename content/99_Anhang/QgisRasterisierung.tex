\subsection{Implementierung der Rasterisierung mit QGIS-Algorithmen}
\label{sec:QgisRasterisierung}

\begin{code}
\begin{minted}[
    linenos,
    numbersep=10pt,
    gobble=0,
    frame=lines,
    framesep=2mm]{python}
import processing
from qgis.core import *

def generate_heatmap(gpkg_path, date, extent, grid_cell_size=4000):
    db_uri = 'postgres://dbname=\'speedcam_archive\' host=localhost ' + \
        'port=5432 user=\'speedcam\' password=\'speedcam\' sslmode=allow ' + \
        'key=\'id\' srid=4326 type=Point checkPrimaryKeyUnicity=\'1\' ' + \
        'table="public"."speedcam" (position) sql=created_date::date = '

    if os.path.isfile(gpkg_path):
        heatmap = QgsVectorLayer(gpkg_path, '', 'ogr')
        return heatmap

    date_str = date.strftime("'%Y-%m-%d'")
    print(f"Processing date {date_str}", flush=True)
    
    grid_layer = processing.run('qgis:creategrid', {
        'EXTENT' : extent,
        'HSPACING' : grid_cell_size, 'VSPACING' : grid_cell_size,
        'CRS' : QgsCoordinateReferenceSystem('EPSG:3857'),
        'HOVERLAY' : 0, 'VOVERLAY' : 0,
        'OUTPUT' : 'TEMPORARY_OUTPUT', 'TYPE' : 2})['OUTPUT']

    heatmap = processing.run('qgis:joinbylocationsummary', {
        'INPUT' : grid_layer,
        'JOIN' : db_uri + date_str,
        'JOIN_FIELDS' : ['duration_hours'],
        'PREDICATE' : [0],
        'SUMMARIES' : [5],
        'OUTPUT' : 'TEMPORARY_OUTPUT',
        'DISCARD_NONMATCHING' : False})["OUTPUT"]

    store_layer(heatmap, gpkg_path)
    return heatmap
\end{minted}
\captionof{listing}{Rasterisierung der Datenpunkte an einem bestimmten Datum}
\label{lst:GenerateHeatmapFunction}
\end{code}

\clearpage
\begin{code}
\begin{minted}[
    linenos,
    numbersep=10pt,
    gobble=0,
    frame=lines,
    framesep=2mm]{python}
import subprocess
from qgis.core import *
from processing.core.Processing import Processing

def gpkg_to_tif(gpkg_path, tif_path, heatmap, grid_cell_size=1000):
    extent = heatmap.extent()
    extent = [round(extent.xMinimum()), round(extent.xMaximum()),
        round(extent.yMinimum()), round(extent.yMaximum())]
    raster_width = round((extent[1] - extent[0])/grid_cell_size)
    raster_height = round((extent[3] - extent[2])/grid_cell_size)

    # Rasterize heatmap using gdal_rasterize subprocess
    subprocess.call(['gdal_rasterize', '-a', 'duration_hours_sum', '-ts',
        str(raster_width), str(raster_height), '-a_nodata', '0.0', '-te',
        str(extent[0]), str(extent[2]), str(extent[1]), str(extent[3]),
        '-ot', 'Float16', '-of', 'GTiff', gpkg_path, tif_path],
        stdout=subprocess.DEVNULL, stderr=subprocess.DEVNULL)

def process_dates(dates):
    QgsApplication.setPrefixPath('/usr', True)
    qgs = QgsApplication([], False)
    qgs.initQgis()
    Processing.initialize()

    grid_cell_size = 4000
    # 200x200 km centered in Baden-Württemberg
    extent = '927792.5265,1127792.5265,6113386.3009,6313386.3009 [EPSG:3857]'
    heatmaps_path = os.path.join(os.path.dirname(__file__), 'heatmaps')

    for date in dates:
        base_path = os.path.join(heatmaps_path,
            'heatmap_' + date.strftime('%Y-%m-%d'))
        gpkg_path = base_path + '.gpkg'
        tif_path = base_path + '.tif'

        heatmap = generate_heatmap(gpkg_path, date, extent, grid_cell_size)
        gpkg_to_tif(gpkg_path, tif_path, heatmap, grid_cell_size=grid_cell_size)
        os.remove(gpkg_path)
\end{minted}
\captionof{listing}{Umwandlung einer \acrshort{gpkg}-Datei in eine GeoTIFF-Datei mit \emph{gdal\_rasterize} und high-level Ablauf der Rasterisierung}
\label{lst:ProcessDatesFunction}
\end{code}

\clearpage
\begin{code}
\begin{minted}[
    linenos,
    numbersep=10pt,
    gobble=0,
    frame=lines,
    framesep=2mm]{python}
import datetime
from joblib import Parallel, delayed

def main():
    start_date = datetime.date(2014, 7, 17)
    end_date = datetime.date(2021, 10, 25)
    dates = [start_date + datetime.timedelta(days=x)
        for x in range(0, (end_date - start_date).days)]

    # Make batches of 50 dates and include the rest in the last batch.
    batch_size = 50
    batches = [dates[i:i + batch_size]
        for i in range(0, len(dates), batch_size)]

    rest_of_dates = [date for date in dates
        if date not in np.array(batches, dtype=object).flatten()]
    batches.append(rest_of_dates)

    # Use parallel processes to speed up the process
    Parallel(n_jobs=24)(delayed(process_dates)(batch)
        for batch in batches)
\end{minted}
\captionof{listing}{Parallele Rasterisierung des Datensatzes}
\label{lst:RasterisierungMainFunction}
\end{code}