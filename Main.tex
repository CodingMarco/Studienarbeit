% !TeX root = ./Main.tex
% !TeX engine = xetex
% !BIB program = bibtex

\documentclass[oneside, openbib, parskip = full-]{scrartcl}
% \includeonly{ % #INFO: wenn man nur bestimmte includes nutzen will
%     content/Anhang_KIS
%     }





\def\myStudentenname{Marco von Rosenberg}
\def\myTitle{Spatio-Temporal Predictions am Beispiel von Geschwindigkeitskontrollen}
\def\myDokumententyp{Studienarbeit}
%\def\myBachelorart{Bachelor of Science (B.Sc.)} % #TODO aus kommentieren wenn keine Bachelorarbeit
\def\myStudiengang{Informatik/Informationstechnik}
\def\myHochschule{Dualen Hochschule Baden-Württemberg Stuttgart}
\def\myMatrikelnummer{9594981}
\def\myKurs{TINF19IN}
\def\myAusbildungsfirma{Keysight Technologies, Böblingen}
\def\myErstgutachter{Johannes Staib}
%\def\myZweigutachter{Vorname Nachname} % #TODO aus kommentieren wenn es keinen gibt
\def\myTitleHeaderImage{content/images/DHBW_Logo}
\def\myOrt{Stuttgart}
\def\Bearbeitungszeitraum{12 Wochen} % #TODO: Bearbeitungszeitraum ändern % #INFO: allgemeine Werte für die Arbeit

%%%
%%% KOMA related
%%%

\usepackage[
    paper=a4paper,
    left=25mm,
    right=25mm,
    top=25mm,
    bottom=20mm,
    footskip=10mm,
    includefoot,
    bindingoffset=0mm
]{geometry}

\usepackage[utf8]{inputenc} % encoding
\usepackage[T1]{fontenc} % selecting font encodings
\usepackage[ngerman]{babel} % German hyphenation
\usepackage[babel,german=quotes]{csquotes} % German quotes
\usepackage{lmodern} %MOST IMPORTANT PACKAGE!!! High resolution font
\renewcommand*\familydefault{\sfdefault} %% für Serifen freie Schrift
\usepackage{textcomp} % für in line {\textmu}

\usepackage{scrlayer-scrpage}
\clearpairofpagestyles
\setkomafont{pageheadfoot}{\sffamily\footnotesize}
\setkomafont{pagehead}{\bfseries}
\setkomafont{pagination}{}

\KOMAoptions{
    headsepline = true,
    footsepline = false,
    plainfootsepline = false
}
\renewcommand*{\sectionmarkformat}{}

\automark{section}
\ihead{\myStudentenname}
\chead{\myDokumententyp}
\ohead{\headmark}
\ofoot*{\pagemark}


\usepackage{scrdate} % Für Datum und Wochentag

%%%
%%% not KOMA related
%%%

\usepackage{relsize}
\def\Cpp{C\kern-.1em\raise.30ex\hbox{\smaller{++}}
\spacefactor1000}  % Für ein schönes "C++", relsize hierfür benötigt.
\usepackage[onehalfspacing]{setspace}

%\usepackage{booktabs} %Never, ever use vertical rules! Never use double rules!
\usepackage{multirow}
\usepackage{comment}

%\usepackage{etoolbox} %use with care! powerful but not can create difficult errors! extends build time
%\AtBeginEnvironment{table}{\sffamily}
%\AtBeginEnvironment{tabular}{\sffamily}

%\usepackage{enumitem}
\usepackage{paralist} % for \begin{compactitem}

% Much much better listings
\usepackage[newfloat, outputdir=aux]{minted}
\newenvironment{code}{\captionsetup{type=listing}}{}
\SetupFloatingEnvironment{listing}{name=Codeausschnitt}

\usepackage{listings}
\lstset{
    basicstyle=\footnotesize, % Global Code Style
    numbers=left,
    numberstyle=\tiny,
    columns=flexible,
    breaklines=true,
    postbreak=\mbox{\textcolor{red}{$\hookrightarrow$}\space},
}
% from https://tex.stackexchange.com/questions/303465/higher-asterisks-in-lstlisting-environment
\makeatletter
\lst@CCPutMacro
    \lst@ProcessOther {"2A}{\raisebox{2pt}{*}}
    \@empty\z@\@empty
\makeatother

\usepackage{etex}
\bibliographystyle{myIEEEtran}

\usepackage{graphicx}
%\usepackage{svg} %no reason to use this (unless you know how to use it)
\usepackage{float}
\usepackage[abs]{overpic} % Für Unterschrift

\usepackage[binary-units = true ]{siunitx}
%\sisetup{scientific-notation = true} % Wissenschaftliche Notation
\sisetup{exponent-product = \cdot}
\sisetup{output-product = \cdot} % Das Punkt statt X
\sisetup{binary-units=true}
\sisetup{output-decimal-marker = {,}}
\sisetup{detect-all} 	% Einstellung das Front für SI gleich Standardfront
\sisetup{per-mode = symbol}
\sisetup{locale=DE}
\usepackage{eurosym}
\DeclareSIUnit{\sieuro}{\mbox{\euro}}

% verwende gerade kein Mathe \usepackage{amsmath, amsthm, amssymb}
\usepackage{amsmath, amssymb}

%\usepackage{booktabs}
\usepackage{graphicx}
\usepackage{adjustbox}
\usepackage{subcaption}
%\usepackage{lscape}
%\usepackage{tikz}
\usepackage{pgfplots}
\pgfplotsset{compat=1.16}
\usepackage{pgfplotstable}

%% Glossar
\usepackage[acronym,nonumberlist,nopostdot]{glossaries}
\newcommand\acrfullr[2][]{\acrshort[#1]{#2} (\acrlong[#1]{#2})}
\usepackage{glossary-mcols} % für Style mcolindex
\glsenablehyper % Verlinkung der Einträge im Text zu Abkürzungen
\setglossarystyle{mcolindex}
\renewcommand{\glossarypreamble}{\glsfindwidesttoplevelname[\currentglossary]} % funktioniert nicht mit mcolindex
\makenoidxglossaries


\usepackage{makecell} % Für multiline Zellen mit \makecell
%\usepackage{tabularx}
\usepackage{longtable}


\usepackage[
    hidelinks,              % Das Links nicht farbig
    bookmarks,              % Bookmarks erstellen
    bookmarksopen,          % Bookmarks bei öffnen des PDF anzeigen
    bookmarksnumbered,      % damit Nummern in Bookmarks
    bookmarksopenlevel = 1, % damit nur die erste Ebene der Bookmarks beim öffnen angezeigt werden
    pdftoolbar=		false,
    pdfmenubar=		false,
    pdftitle={		\myDokumententyp~über~\myTitle~von~\myStudentenname},
    pdfsubject={}
    pdfkeywords={}
    pdfauthor={		\myStudentenname},
    pdfstartpage={	1}
    ]{hyperref}
\addto\extrasngerman{\def\subsectionautorefname{Abschnitt}}
\addto\extrasngerman{\def\subsubsectionautorefname{Abschnitt}}
\addto\extrasngerman{\def\appendixautorefname{Anhang}} % funktioniert auch nicht
\addto\extrasngerman{\def\lstlistingautorefname{Codeausschnitt}}
\renewcommand{\lstlistingname}{Codeausschnitt}

%% Anhang
\usepackage{appendix}
\renewcommand{\appendixname}{Anhang}
\renewcommand{\appendixtocname}{Anhang}
\renewcommand{\appendixpagename}{Anhang}
%\newcommand{\appendixautorefname}{Anhang} % funktioniert nicht
\newcommand*\appendixmore{% see the KOMA-Script documentation
\clearpage
%\addsec{{\appendixname}}%
\renewcommand{\thesubsection}{{\Alph{subsection}}}%
} % damit Appendix subsection mit Buchstaben anfangen

\newcommand{\appendixref}[1]{\hyperref[#1]{Anhang~\ref{#1}}}

\renewcommand{\k}[1]{\textit{#1}}
\newcommand{\f}[1]{\textbf{#1}}
\let\oldautoref\autoref
\let\oldref\ref

\renewcommand{\autoref}[1]{\oldautoref{#1}}
\renewcommand{\ref}[1]{\oldref{#1}}
\newcommand{\fullref}[1]{\autoref{#1} \nameref{#1}}
  % #INFO: Einstellungen
%\newglossaryentry{control endpoint}{name={control endpoint}, description={logische, addressierte Schnittselle in \acrshort{usb} }}
%\newglossaryentry{ieee488}{name={IEEE-488},description={}}

% Abkürzungen
\newacronym{nn}{NN}{\k{neuronales Netz}}
\newacronym{relu}{ReLU}{\k{rectified linear unit}}
\newacronym{mse}{MSE}{\k{mean squared error}}
\newacronym{rnn}{RNN}{\k{rekurrentes neuronales Netz}}
\newacronym{lstm}{LSTM}{\k{Long Short Term Memory}}
\newacronym{cnn}{CNN}{\k{convolutional neural network}}
 % #INFO: laden aller Glossary und Abkürzungen

\begin{document}
%%%
%%% Vorbau
%%%
\pagenumbering{Roman}
\begin{titlepage}
	\thispagestyle{empty}
    \begin{center}
        \rule{\textwidth}{1.6pt}\vspace*{-\baselineskip}\vspace*{2pt}
        \rule{\textwidth}{0.4pt}
        \\[0.2\baselineskip]
        {\LARGE \myTitle}	
        \rule{\textwidth}{0.4pt}\vspace*{-\baselineskip}\vspace{3.2pt}
        \rule{\textwidth}{1.6pt}
        \\
        \vspace*{12mm}	{\large \myDokumententyp}\\
        \ifx\myBachelorart\undefinde
            \vspace*{27mm}	Studienganges \myStudiengang\\
        \else
            \vspace*{12mm}	für die Prüfung zum\\
            \vspace*{3mm} 	{\large \myBachelorart}\\
            \vspace*{12mm}	Studienganges \myStudiengang\\
        \fi
        \vspace*{3mm} 	an der \myHochschule\\
        \vspace*{12mm}	von\\
        \vspace*{3mm} 	{\large \myStudentenname}\\
        \vspace*{12mm}	{\todaysname} der {\today} \\
      \end{center}

      \vfill
      \begin{spacing}{1.2}
        \centering
        \begin{tabbing}
            \hspace{6.5cm}                   \= \kill\\
            \textbf{Matrikelnummer, Kurs}  \>  \myMatrikelnummer, \myKurs\\
            \textbf{Ausbildungsfirma}      \>  \myAusbildungsfirma\\
            \textbf{Erstgutachter}         \>  \myErstgutachter\\
            \ifx\myZweigutachter\undefined
            \else
            \textbf{Zweitgutachter}        \>  \myZweigutachter\\
            \fi
            \textbf{Bearbeitungszeitraum}  \>  \Bearbeitungszeitraum\\
        \end{tabbing}
      \end{spacing}
\end{titlepage}	
\section*{Selbständigkeitserklärung}
\vspace{1cm}
\begin{tabbing}
    \hspace{.175\textwidth} \= \hspace{.825\textwidth}\=\kill\\
    Name:		            \> \myStudentenname\\
    Matrikelnummer:		    \> \myMatrikelnummer\\
    Studiengang:		    \> \myStudiengang\\
    Kurs:				    \> \myKurs\\
    Titel:	                \> \myTitle             \>\\
\end{tabbing}
\vspace{1cm}
Ich versichere hiermit, dass ich die vorliegende Arbeit selbstständig verfasst und
keine anderen als die angegebenen Quellen und Hilfsmittel benutzt habe.\\
Falls sowohl eine gedruckte als auch elektronische Fassung abgegeben wurde,
versichere ich zudem, dass die eingereichte elektronische Fassung mit der gedruckten
Fassung übereinstimmt.\\
\vspace{2cm}\\
\begin{minipage}{0.99\textwidth}
	\centering
	\begin{minipage}[t]{0.5\textwidth}
	\hspace{2cm}
		\begin{tabular}{c}
			\myOrt, {\today}\\
			\\
			\\
		\end{tabular}
	\end{minipage}
	\hfill
	\begin{minipage}[t]{0.4\textwidth}
		\begin{tabular}{c}
			{\myStudentenname}\\
			\\
			\\
		\end{tabular}
	\end{minipage}
\end{minipage}

\section*{Abstract}
\#TODO: Abstract
\section*{Kurzfassung}
\#TODO: Kurzfassung

% --- Inhaltsverzeichnis ---
\pdfbookmark[section]{\contentsname}{toc}
\tableofcontents

\clearpage
% --- Abbildungsverzeichnis --
\phantomsection
\addcontentsline{toc}{subsection}{Abbildungsverzeichnis}
\listoffigures

% --- Tabellenverzeichnis ---
\phantomsection
\addcontentsline{toc}{subsection}{Tabellenverzeichnis}
\listoftables

% --- Listing Verzeichnis --- % #INFO: Verwende ich gerade nicht
\phantomsection
\addcontentsline{toc}{subsection}{Codeausschnittsverzeichnis}
\lstlistoflistings

% --- Glossar ---
\iffalse % #INFO: auf \iftrue setzen falls man ein Glossar will
\phantomsection
\addcontentsline{toc}{subsection}{Glossar}
\printnoidxglossary[sort=word,title=Glossar] % main glossary
\fi

% --- Abkürzungen ---
\phantomsection
\addcontentsline{toc}{subsection}{Abkürzungen}
\printnoidxglossary[type=acronym,title=Abkürzungen] % Abkürzungen


\clearpage

%%%
%%% Inhalt
%%%
\pagenumbering{arabic}
% #INFO: für Inhalt normalerweise \input. \include with care, kann Verweise und so schwieriger machen
% ich nutze trotzdem include damit ich mit \includeonly{} zum test bauen kleiner PDFs erstellen lassen kann
%\input{content/<Name des Kapitels>}
%\include{content/<filename>}

\section{Vorlage}
\subsection[USB]{\acrshort{usb}}
\acrfull{usb} ist ein Schnittstelle zur Anbindung von \acrshort{usb}-Geräten.


\subsubsection{Zitieren} \label{sssec:zitieren}

Zitieren \cite[s. 31]{usb_developer_guide} angeschlossen werden, die sich auf Stern-Sternnetzten, mit zu 7 Ebenen \cite[4.1.1]{usb_developer_guide}, aufteilen können. Der Steckplatz von einem Hub zum Host, als upstream port bezeichnet, Buchsen für Peripheriegeräten oder weiter Hubs als downstream ports.
Die Struktur einer logischen Verbindung von Host zu einem Slave ist in \autoref{fig:usb_pipe_modell} 
\begin{figure}[h]
    \centering
    \includegraphics[width=0.75\textwidth,height=6cm,keepaspectratio=true]{content/images/Cypress_ubs_figure_1.PNG}
    \caption{\acrshort{usb} pipe Modell \cite[Figure~1]{usb_developer_guide}}
    \label{fig:usb_pipe_modell}
\end{figure}
\par
Ein endpoints kann entweder die erwähnten Protokollfunktionen bereitstellen, Daten vom Host zu empfangen, dann ist dies ein IN endpoint, oder Daten für den Host bereitzustellen dann ist dies ein OUT endpoint. Diese Terminologie folgt der in der USB Spezifikation verwendete Betrachtung des Bus aus Sicht des Hosts.
Neben der Richtung eines endpoints, IN oder OUT, werden vier Übertragungsarten \cite[5.4]{usb_developer_guide} unterschieden. Die Übertragungsart wird durch die Gerätekonfiguration für jeden endpoint festleget und wird in \autoref{sssec:endpoints} weiter ausgeführt. Die Funktionen der Übertragungsarten sind \cite[4.7]{usb_developer_guide}:

\begin{compactitem}
    \item \textbf{control transfer} Für das Übertragen von Konfigurations- und Steuerungsbefehlen über die control pipe.
    \item \textbf{interrupt transfer} Für die Übertragung kleiner Datenmengen mit einer möglichst kleinen Latenz über eine data pipe .
    \item \textbf{bulk transfer} Für die Übertragung grö{\ss}er Datenmengen ohne zugesicherten Geschwindigkeit oder Latenz über eine data pipe ohne Datenverluste.
    \item \textbf{isochronous transfer}  Für Übertragungen mit zugesicherten Übertragungskapazität über eine data pipe ohne Fehlerbehebung.
\end{compactitem}
Welche Übertragungsarten ein Gerät unterstützt, und damit über welche endpoints es verfügt kann frei gewählt werden oder kann einer %\acrshort{ubs} Klassen Spezifikation folgen. Eine solche Spezifikation definiert neben dem Aufbau der endpoints weiter einheitliche Funktionen damit Geräte, einer gleichen Klasse, problemlos ausgetauscht werden können.

Softwareseitig wird die Übertragung in drei Ebenen unterteilt, diese sind in \autoref{fig:usb_schnittstellen_abstraktion} abgebildet.
\begin{figure}[h]
    \centering
    \includegraphics[width=.75\textwidth,keepaspectratio=true]{content/images/usb_spec_figure_5_2.PNG}
    \caption{\acrshort{usb} Schnittstellen Abstraktion \cite[Figure~5-2]{usb_developer_guide}}
    \label{fig:usb_schnittstellen_abstraktion}
\end{figure}



\subsubsection{Endpoints} \label{sssec:endpoints}
Nach  \acrshort{usb} Spezifikation ist ein endpoint eine eindeutiger adressierbarer Schnittstelle eines \acrshort{usb} Gerätes der Daten senden oder empfangen werden. Ausgenommen ist der control endpoint, dessen Verwendung wird in \autoref{sssec:Konfigurationsvorgang} ausführlicher beschrieben.

Die \acrshort{usb} Spezifikation definiert vier Arten von endpoints mit der maximalen Paketgrö{\ss}e und den unterstützen Übertragungsgeschwindigkeiten. Die Art eines endpoints definiert gleichzeitig die die \k{transfer}-Art aus \fullref{sssec:usb_ueberblick} und sind \cite[Kapitel~8]{usb_developer_guide}:
\begin{compactitem}
    \item \textbf{control endpoint} für control Transfers und muss in jedem Gerät vorhanden sein, um Geräteinformation zu übertragen. Die Übertragung sind fehlerbehebend und können, bei High-Speed, bis zu 10\,\% der Buslast einnehmen.
    \item \textbf{control endpoint} für control Transfers und muss in jedem Gerät vorhanden sein, um Geräteinformation zu übertragen. Die Übertragung sind fehlerbehebend und können, bei High-Speed, bis zu 10\,\% der Buslast einnehmen.
\end{compactitem}

Im  device descriptor werden allgemeine Informationen für ein Gerät bereitgestellt, die Datenfelder sind in \autoref{tab:ubs_device_descriptor} aufgetragen.
\begin{table}[h]
    \centering
    \begin{tabular}{|l|l|}
        \hline
        \textbf{Feld}      & \textbf{Bedeutung}                                 \\
        \hline
        bLength            & Länge des descriptor               \\
        \hline
        bDescriptorType    & descriptor Type, hier DEVICE (0x01)                 \\
        \hline
        bcdUSB             & \acrshort{usb} Spezifikationsversion als \acrshort{bcd}                   \\
        \hline
        bDeviceClass       & device class                                       \\
        \hline
        bDeviceSubClass    & device subclass                                    \\
        \hline
        bDeviceProtocol    & Device Protocol                                    \\
        \hline
        bMaxPacketSize     & Maximale Packet Größe für endpoint 0               \\
        \hline
        idVendor           & \acrfull{vid}, vergeben durch \acrshort{usbif})        \\
        \hline
        idProduct          & \acrfull{produkt_id}, vergeben durch den Hersteller   \\
        \hline
        bcdDevice          & Geräteversion als \acrshort{bcd}                              \\
        \hline
        iManufacturer      & Position der Herstellerbezeichnung in Packet       \\
        \hline
        iProduct           & Position der Gerätebezeichnung in Packet           \\
        \hline
        iSerialNumber      & Position der Seriennummer in Packet                \\
        \hline
        bNumConfigurations & Anzahl der unterschiedlichen Gerätekonfiguration   \\
        \hline
    \end{tabular}
    \caption{device descriptor}
    \label{tab:usb_device_descriptor} % #TODO Offset und Size entfernen
\end{table}



    %\include{content/Schluss}

%%%
%%% Nachbau
%%%
\addsec{Literatur}
\renewcommand{\section}[2]{}
\bibliography{citation/quellen}
\clearpage

% Anhänge
\appendix
\addappheadtotoc
\appendixpage
\begin{appendices}
%\include{content/<Name des Anhangs>}
\newcommand*{\tabletype}{longtable}

\subsection[KIS Requirements]{Requirements }\label{app:anhang_kis_requirements}
\subsubsection{Instrument Hardware ID Requirements}
\begin{\tabletype}[H]{|p{.12\linewidth}|p{.83\linewidth}|}
\hline
\textbf{Nr.} & \textbf{Requirement} \\ \hline 
\endhead 
4.1-2.3.1 & Existing  devices MAY continue to use the existing JTAG or JEDEC manufacturer ID \\ \hline 
4.1-2.3.2 & New  JEDEC devices (released after August 1, 2014) MUST use the  JEDEC manufacturer ID \\ \hline 
\caption {Instrument Hardware ID Requirements}
\label{tab:InstrumentHardwareIDRequirementsList}
\end{\tabletype}


\end{appendices}
\end{document}
